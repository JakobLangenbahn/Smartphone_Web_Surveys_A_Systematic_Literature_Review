\documentclass[12pt,a4paper]{article}

% Import Packages
\usepackage[utf8]{inputenc}
\usepackage{graphicx}
\usepackage{subfiles}
\usepackage[style = apa]{biblatex}
\usepackage[english]{babel}
\usepackage{csquotes}
\usepackage{hyperref}
\usepackage{url}
\usepackage{doi}
\usepackage{booktabs}
\usepackage{rotating}
\usepackage{lscape}
\usepackage{longtable}
\usepackage{float}

% Import the bibliography file
\addbibresource{export.bib} 

\begin{document}

% Cover Page
\begin{titlepage}
   \begin{center}
   
       \vspace*{1cm}
       
       \large
       \textbf{Smartphone Web Surveys: A Systematic Literature Review}
        
       \vspace{1.5cm}
   
       \textbf{1563986}
   
       \vfill
   
       A term paper presented for the lecture\\
       Data and Measurement
   
       \vspace{0.8cm}
   
       \includegraphics[width=0.4\textwidth]{reports/figures/university_logo.png}
       
       University of Mannheim\\
       Germany\\
   
       \date{\today}
   
   \end{center}
\end{titlepage}


% Title and Abstract
\thispagestyle{plain}
\begin{center}

    \Large
    \textbf{Smartphone Web Surveys: A Systematic Literature Review}
     
    \vspace{0.4cm}

    \today
    
    \vspace{0.9cm}
    
    \abstract{Online surveys are a dominant source of survey data that is increasingly assessed with mobile devices, especially smartphones. Thus there is a wide range of research focusing on various aspects of smartphone participation in these surveys. There are already different reviews that focus on a specific branch of the field, but no literature reviews give an overview of the whole research field. This work closes the gap 
    by employing the methodology of systematic literature reviews to identify relevant research objectively. For this, we identified and analyzed 86 relevant research articles from peer-reviewed journals or conference proceedings from 2008 to 2021. We identified the increasing trend of publication in the field until 2019 and separated our detailed overview in survey quality and survey design. We identified several research gaps while the most crucial research gaps in the missing coverage of the population of different world regions. Most research is based on survey data from western Europe and North America. This observation induces the need for future research projects that translate existing research to other regions and update their results. Since a graduate student wrote this course work, we identified several weaknesses that need to be minded when interpreting the results.}
    
    \vspace{0.4cm}
    
    \noindent
    \textbf{Keywords:} Smartphone; Web Survey; Survey Data Quality; Mobile Survey Design; Systematic Literature Review
    
\end{center}

\newpage


% Include Content

\section{Introduction}
\label{01_introduction}
\subfile{sections/01_introduction}

\section{Literature Review Methodology}
\label{02_literature_review_methodology}
\subfile{sections/02_literature_review_methodology}

\section{Analysis Results}
\label{03_analysis_results}

\subsection{General Development Analysis}
\label{03_01_general_development_analysis}
\subfile{sections/03_01_general_development_analysis}

\subsection{Survey Quality Analysis}
\label{03_02_survey_quality_analysis}
\subfile{sections/03_02_survey_quality_analysis}

\subsection{Survey Design Analysis}
\label{03_04_survey_design_analysis}
\subfile{sections/03_03_survey_design_analysis}

\section{Summary and Outlook}
\label{04_summary_and_outlook}
\subfile{sections/04_summary_and_outlook}

% Include Bibliography
\newpage
\printbibliography

\newpage

I/we declare that this assignment is my/our own original piece of work. I/we have not copied from other sources without citing these, submitted part or all of this work as an assignment on another course or plagiarised in any other way. I/we agree that my/our work may be stored electronically and sent to third parties for the purpose of electronic checks on plagiarism.

\begin{figure}[H]
    \centering
    \includegraphics[width=0.5\textwidth]{signature.JPG}
\end{figure}

\end{document}