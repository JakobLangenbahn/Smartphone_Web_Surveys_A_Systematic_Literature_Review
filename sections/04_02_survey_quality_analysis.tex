Simple meta-analysis of survey data comparing different years is not possible based on a tremendous shift in the smartphone and mobile phone enviroemnt, this lead to the effect that the population changed and the gegenebenheiten

In this seciton we are going to analyse the research on data quality in smartphone web surveys. One has to differentiate between to effects. One is the mode effect of a web survey compared to other modes (CAPI, CATI etc). And then the effect of the device mobile versus pc (theoretically smartphonw versus tablet or feature phone). We will concentrate on the diffeerence on data quality between pc and smartphone in web survey. The intereseted reader can reffer to [citation] for a comparison of different modes. We will show the research fields in this and make an non quantiative summary of the most investigated items in data quality. 

There are various topics discussed in the reviewed paper, where 34 paper have data quality as a main topic. There already exist reviews to the data quality [citation]. We will update these review and complement them with this analysis. 

 'mobile design not optimized mobile avoid text answer more than optimized',
 'mobile design not optimized mobile completion time higher than optimized',
 'mobile design not optimized mobile less text than optimized',
 'mobile design not optimized mobile non substantial anwser more than optimized',
 mobile design optimized online higher completion rate
 mobile design optimized online higher completion rate

\begin{table}
	\centering
	\begin{tabular}{ll}
		\toprule
		Data Quality Topic  &  Research Articles \\
		\midrule
        mobile acquiescence more    & 0  \\
        mobile anwser changes more    & 0  \\
        mobile bias   & 0  \\
        mobile breakoff rate higher    & 0  \\
        mobile check all that apply anwsers less    & 0  \\
        mobile completion rate lower    & 0  \\
        mobile completion time higher    & 0  \\
        mobile construct validity    & 0  \\
        mobile device influence
        mobile enjoyment of survey less    & 0  \\
        mobile error rate higher    & 0  \\
        mobile estimates biased more    & 0  \\
        mobile inconsistent anwsers more    & 0  \\
        mobile instruction manipulation check failure more    & 0  \\
        mobile inter item correlation grid more    & 0  \\
        mobile lost focus more    & 0  \\
        mobile meaningfull answers less    & 0  \\
        mobile measurement quality lower    & 0  \\
        mobile midpoint responding more    & 0  \\
        mobile missing items rate higher    & 0  \\
        mobile more heaping    & 0  \\
        mobile more primacy    & 0  \\
        mobile more socially undesirables anwsers    & 0  \\
        mobile motivated underreporting more    & 0  \\
        mobile non differentiated answers more    & 0  \\
        mobile non substantial anwser more    & 0  \\
        mobile primacy effect higher    & 0  \\
        mobile reponse rate lower    & 0  \\
        mobile response time faster    & 0  \\
        mobile results different    & 0  \\
        mobile satisfaction lower    & 0  \\
        mobile satisficing check box more    & 0  \\
        mobile scale effects more    & 0  \\
        mobile sensitive questions answering less    & 0  \\
        mobile sensitive questions less sensitive    & 0  \\
        mobile socially undesirable answers more    & 0  \\
        mobile straightlining less    & 0  \\
        mobile technical difficulties more    & 0  \\
        mobile text answer avoid more    & 0  \\
        mobile text answer shorter    & 0  \\
        mobile text response nonrelevant more    & 0  \\
        mobile unaceptably short anwser less    & 0  \\
        mobile usability lower    & 0  \\
        mobile willingness rate lower & 0 \\
		\bottomrule 
	\end{tabular}
	\caption{Overview of the Professional Survey Operators used for more than one survey in the review}
	\label{tab: author}
\end{table}

There is a lot of research going on data quality as we could see in the previous table.. If we compare this with the quality indicators in (Citiation of an overview of data quality indicators) we see that currently research on is misisng. as this is more complciuated more sophistiacted tehcnolgies would need to be used. 

O

Further analyses could be on
- Straightlining
- Primacy effect
- Non Substantial Answer
- Text Anwser


We have to view this results as temporary snapshot of the behavior, as our behaviour around mobile device is changing based on a rapidly developing development of the technology with ever new helps, large screen size and more learning effects for various groups (children, teenager, Adults, older population. As well as new design insights, that help to reduce the negative effext and coild also increase positive effects of mobile technologies.

One could go into more detailed in this section and do quantiatvie meta analysies to synthesize the different research results. This could be an future research endevaour. 


In this section we analyze the participants and possible participants of mobile surveys. There are two important thinks here to consider. One ist see who is taking surveys on their mobile phone in the moemnt and if there is a ubstantial bias induced by this. Second we check who is willing to take surveys on a smartphone for future endevaours. Both is helpful to understand problems like coverage bias and simialr, for conducting mobile surveys.

We will start with a small overview of the research areas that are present in this work about the participants. After this we will nanalyze specific point and aggregate the results in a qualitative way.



\begin{table}
	\centering
	\begin{tabular}{ll}
		\toprule
		Data Quality Topic  &  Research Articles \\
		\midrule
        mobile participants coverage bias \& mobile participants representative not    & 0  \\
        mobile participants education    & 0  \\
        mobile participants income higher   & 0  \\
        mobile participants less alone    & 0  \\
        mobile participants less trust in confidentiality    & 0  \\
        mmobile participants millenial behavior    & 0  \\
        mobile participants more distracted    & 0  \\
        mobile participants more female
        mobile participants more mobile    & 0  \\
        mobile participants more moving    & 0  \\
        mobile participants more multitasking    & 0  \\
        mobile participants more progressive    & 0  \\
        mobile participants more urban    & 0  \\
        mobile participants more willing passive data collection    & 0  \\
        mobile participants more stranger present    & 0  \\
        mobile participants multiple browser sessions more    & 0  \\
        mobile participants sensitivity perceived higher    & 0  \\
        mobile participants social class higher    & 0  \\
        mobile participants younger    & 0  \\
		\bottomrule 
	\end{tabular}
	\caption{Overview of the Professional Survey Operators used for more than one survey in the review}
	\label{tab: author}
\end{table}




There are a lot more dimensions that we could possible analysises given more time and more resources. An systematic review only on participants would be an interesting research project. Especially the changes in this will be interesting depending on future channgees in the use of smartphones espeically for elderly [citation.]

Missing how to motivate and keep people in surveys on mobile devices

invitaiton modes as push messages