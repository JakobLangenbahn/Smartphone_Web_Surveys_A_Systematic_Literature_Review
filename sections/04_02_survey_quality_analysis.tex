Which dimension of survey quality of smartphone participation in web surveys are investigated, and which factors lack qualitative research?

Survey Quality as evaluated by the total survey error framework \cite{groves_total_2010} or other measures it important to access the validity of survey results. In this section we will describe the quality dimension covered in the articles covered in our literature review and search for potential research gaps. This will be separated in two parts, in the first one concentrates on quality indicators that compare the impact of the use of smartphones instead of personal computers in web surveys and the seconds covers research on the mobile participants of web surveys.

When analysing the data quality of mobile web surveys, a significant majority of research articles evaluates the quality by comparing the measurement objects with the results of personal computer user. The underlying assumption is that for comparing with other modes one can refer to the existing results comparing web surveys to to other modes (as for example \cite{manfreda_web_208, ryan_survey_2020, evans_the_2018}). We will not analyse the difference between smartphone participants and other mobile devices like tablets, as this is out of scope of this work and there is no consensus if tablet are more similar to smartphones or personal computers \cite{couper_why_2017}. Our qualitative overview of research topics, can be found in table \ref{tab: quality}. A next potential research project would be to execute quantitative meta analyses on topics covered by enough research articles. This is out of scope for this work, but a first analyses showed interesting results. 


\begin{table}
	\centering
	\begin{tabular}{ll}
		\toprule
		Data quality category & Articles in our review\\
		\midrule
    	Completion Rate & \cite{ha_are_2019, ha_data_2020}\\
    	& \cite{mavletova_data_2013, mavletova_grid_2018}\\
    	& \cite{ buskirk_making_2014, mavletova_sensitive_2013}\\
        Known Manipulation Bias & \cite{wells_comparison_2014, hartman_does_2019}\\
    	& \cite{peytchev_experiments_2010, tourangeau_web_2017}\\
    	& \cite{keusch_web_2017}\\
        Completion Time & \cite{revilla_are_2017, ha_are_2019}\\
    	& \cite{revilla_are_2017, de_bruijne_comparing_2013}\\
    	& \cite{weigold_computerized_2021, ha_data_2020}\\
    	& \cite{mavletova_data_2013, hartman_does_2019}\\
    	& \cite{liebe_does_2015, antoun_effects_2017}\\
    	& \cite{lee_experimental_2019, gummer_explaining_2015}\\
    	& \cite{antoun_factors_2020, mavletova_grid_2018}\\
    	& \cite{lambert_living_2015, buskirk_making_2014}\\
    	& \cite{schlosser_mobile_2018, zou_mobile_2021}\\
    	& \cite{daikeler_motivated_2020, revilla_open_2016}\\
    	& \cite{skeie_smartphone_2019, revilla_testing_2018}\\
    	& \cite{huff_comparison_2015, mason_effect_2019}\\
    	& \cite{struminskaya_effects_2015, lugtig_use_2016}\\
    	& \cite{tourangeau_web_2018, keusch_web_2017}\\
    	& \cite{toepoel_what_2014, couper_why_2017}\\
        Mobile Device Impact & \cite{schlosser_mobile_2018, mavletova_mobile_2014}\\
        Sensitive Questions & \cite{lee_experimental_2019, mavletova_sensitive_2013}\\
    	& \cite{toninelli_smartphones_2016}\\
        Primacy effect & \cite{erens_comparing_2019, wells_comparison_2014}\\
    	& \cite{mavletova_data_2013, lugtig_use_2016}\\
    	& \cite{tourangeau_web_2017, toepoel_what_2014}\\
        Text answer avoiding & \cite{erens_comparing_2019, wells_comparison_2014}\\
    	& \cite{ha_data_2020, antoun_effects_2017}\\
    	& \cite{de_bruijne_improving_2014, schlosser_mobile_2018}\\
    	& \cite{revilla_open_2016, struminskaya_effects_2015}\\
    	& \cite{toepoel_what_2014}\\
        Text answer length & \cite{wells_comparison_2014, mavletova_data_2013}\\
    	& \cite{antoun_effects_2017, lambert_living_2015}\\
    	& \cite{buskirk_making_2014, schlosser_mobile_2018}\\
    	& \cite{zou_mobile_2021, revilla_open_2016}\\
    	& \cite{toepoel_probing_2021, struminskaya_effects_2015}\\
    	& \cite{lugtig_use_2016, toepoel_what_2014}\\
    		\end{tabular}
	\caption{Overview of the Professional Survey Operators used for more than one survey in the review}
	\label{tab: quality_part_1}
\end{table}

\begin{table}
	\centering
	\begin{tabular}{ll}
		\toprule
		Data quality category & Articles in our review\\
		\midrule
        Response rate & \cite{lambert_living_2015, tourangeau_web_2018}\\
        Straightlining & \cite{erens_comparing_2019, mavletova_grid_2018}\\
    	& \cite{lugtig_recruiting_2019, struminskaya_effects_2015}\\
    	& \cite{lugtig_use_2016, tourangeau_web_2018}\\
    	& \cite{keusch_web_2017}\\
        Non substantial Answers & \cite{revilla_comparing_2018, mavletova_data_2013}\\
    	& \cite{antoun_effects_2017, schlosser_mobile_2018}\\
    	& \cite{revilla_open_2016, toepoel_sliders_2018}\\
    	& \cite{skeie_smartphone_2019}\\
        Scale effects & \cite{krebs_exploring_2021, tourangeau_web_2018}\\
        Break-off rate & \cite{erens_comparing_2019, mavletova_data_2013}\\
    	& \cite{hartman_does_2019, liebe_does_2015}\\
    	& \cite{lee_experimental_2019, mavletova_grid_2018}\\
    	& \cite{lambert_living_2015, bosch_measurement_2019}\\
    	& \cite{schlosser_mobile_2018, brosnan_pc_2017}\\
    	& \cite{mavletova_sensitive_2013, keusch_web_2017}\\
    	& \cite{toepoel_what_2014, steinbrecher_why_2015}\\
    	& \cite{couper_why_2017}\\
        Usability & \cite{mavletova_grid_2018, huff_comparison_2015}\\
        Survey Satisfaction & \cite{de_bruijne_comparing_2013 mavletova_grid_2018}\\
    	& \cite{mavletova_sensitive_2013, lugtig_use_2016}\\
    	& \cite{toepoel_what_2014}\\
        Socially Undesirable Answers & \cite{mavletova_data_2013, antoun_effects_2017}\\
        Satisficing & \cite{revilla_experiment_2017, erens_comparing_2019}\\
    	& \cite{revilla_comparing_2018, mavletova_data_2013}\\
    	& \cite{daikeler_motivated_2020, lugtig_recruiting_2019}\\
    	& \cite{toepoel_sliders_2018, struminskaya_effects_2015}\\
    	& \cite{lugtig_use_2016, keusch_web_2017}\\
        Missing items rate & \cite{revilla_are_2017, erens_comparing_2019}\\
    	& \cite{revilla_comparing_2018, lee_experimental_2019}\\
    	& \cite{buskirk_making_2014, daikeler_motivated_2020}\\
    	& \cite{toepoel_sliders_2018, struminskaya_effects_2015}\\
    	& \cite{lugtig_use_2016, tourangeau_web_2018}\\
    	& \cite{keusch_web_2017}\\
        Inconsistent, erroneous & \cite{revilla_comparing_2018, weigold_computerized_2021}\\
    	or random Answers& \cite{hartman_does_2019, bosch_measurement_2019}\\
    	& \cite{antoun_simultaneous_2019, skeie_smartphone_2019}\\
    	& \cite{revilla_testing_2018, huff_comparison_2015}\\
        Response time & \cite{liebe_does_2015, schlosser_mobile_2018}\\
        Other & \cite{revilla_are_2017, revilla_comparing_2018}\\
    	& \cite{de_bruijne_comparing_2013, mavletova_data_2013}\\
    	& \cite{antoun_factors_2020}\\
		\bottomrule 
	\end{tabular}
	\caption{Overview of the Professional Survey Operators used for more than one survey in the review}
	\label{tab: quality_part_2}
\end{table}

There is a lot of research going on data quality as we could see in the previous table.. If we compare this with the quality indicators in (Citiation of an overview of data quality indicators) we see that currently research on is misisng. as this is more complciuated more sophistiacted tehcnolgies would need to be used. 




We have to view this results as temporary snapshot of the behavior, as our behaviour around mobile device is changing based on a rapidly developing development of the technology with ever new helps, large screen size and more learning effects for various groups (children, teenager, Adults, older population. As well as new design insights, that help to reduce the negative effext and coild also increase positive effects of mobile technologies.

One could go into more detailed in this section and do quantiatvie meta analysies to synthesize the different research results. This could be an future research endevaour. 


In this section we analyze the participants and possible participants of mobile surveys. There are two important thinks here to consider. One ist see who is taking surveys on their mobile phone in the moemnt and if there is a ubstantial bias induced by this. Second we check who is willing to take surveys on a smartphone for future endevaours. Both is helpful to understand problems like coverage bias and simialr, for conducting mobile surveys.

We will start with a small overview of the research areas that are present in this work about the participants. After this we will nanalyze specific point and aggregate the results in a qualitative way.



\begin{table}
	\centering
	\begin{tabular}{ll}
		\toprule
		Mobile Participants Topic  &  Research Articles \\
		\midrule
        Coverage Bias & \cite{keusch_coverage_2020, bucher_exploring_2021}\\
        & \cite{fuchs_the_2009}\\
         Education & \cite{de_comparing_2013, gummer_does_2019}\\
        & \cite{de_mobile_2014, zou_mobile_2021}\\
        & \cite{skeie_smartphone_2019, keusch_web_2017}\\
        & \cite{toepoel_what_2014}\\
         Income & \cite{zou_mobile_2021, skeie_smartphone_2019}\\
        & \cite{toepoel_what_2014}\\
         Gender & \cite{de_comparing_2013, wells_comparison_2014}\\
        & \cite{ liebe_does_2015, lambert_living_2015}\\
        & \cite{bosch_measurement_2019, schlosser_mobile_2018}\\
        & \cite{de_mobile_2014, zou_mobile_2021}\\
        & \cite{brosnan_pc,_2017, skeie_smartphone_2019}\\
        & \cite{keusch_web_2017, toepoel_what_2014}\\
         Age & \cite{de_comparing_2013, wells_comparison_2014}\\
        & \cite{gummer_does_2019, liebe_does_2015}\\
        & \cite{bosch_measurement_2019, de_mobile_2014}\\
        & \cite{zou_mobile_2021, brosnan_pc,_2017}\\
        & \cite{skeie_smartphone_2019, bosch_using_2021}\\
        & \cite{keusch_web_2017, toepoel_what_2014}\\
         Urbanity & \cite{de_mobile_2014, toepoel_what_2014}\\
         Distraction & \cite{antoun_effects_2017, toninelli_smartphones_2016}\\
         Presence of other & \cite{antoun_effects_2017, mavletova_sensitive_2013}\\
        & \cite{toninelli_smartphones_2016}\\
         Mobility of participants & \cite{de_comparing_2013, mavletova_data_2013}\\
        & \cite{antoun_effects_2017, mavletova_sensitive_2013}\\
        & \cite{toninelli_smartphones_2016}\\
         Other & \cite{de_mobile_2014, mavletova_sensitive_2013}\\
        & \cite{toninelli_smartphones_2016, keusch_using_2021}\\
        & \cite{toepoel_what_2014}\\
        General usage development & \cite{revilla_do_2016, gummer_does_2019}\\
        & \cite{de_mobile_2014, wells_what_2015}\\
        Willigness to use & \cite{wenz_willingness_2019}\\
        Usage dependig on sex, age etc & \cite{haan_can_2019, keusch_coverage_2020}\\
        & \cite{revilla_do_2016, antoun_simultaneous_2019}\\
        & \cite{keusch_using_2021}\\
        Device use & \cite{keusch_coverage_2020}\\
		\bottomrule 
	\end{tabular}
	\caption{Overview of the Professional Survey Operators used for more than one survey in the review}
	\label{tab: participants}
\end{table}




There are a lot more dimensions that we could possible analysises given more time and more resources. An systematic review only on participants would be an interesting research project. Especially the changes in this will be interesting depending on future channgees in the use of smartphones espeically for elderly [citation.]

Missing how to motivate and keep people in surveys on mobile devices

invitaiton modes as push messages