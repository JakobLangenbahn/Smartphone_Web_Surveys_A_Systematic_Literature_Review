Survey design
- All findings not systematic analysis

In this section we want to analyze the research that is done on survey design for mobile phone web surveys. There were already first reviews on this topics, that include also deisng recommendaation. Therefore we will focus in this section on the an meta overview of the design decision that were already researched and the missing design choices, that could helop icnrease the data quality of mobile devices. For this we can see in figure x the reseachred dim ensions and desing decision.


\begin{table}
	\centering
	\begin{tabular}{ll}
		\toprule
		Data Quality Topic  &  Research Articles \\
		\midrule
         'mobile design agree disagree shorter completion time than item specific question',   & 0  \\
         always on next button   & 0  \\
        android spinne versus radio buttons   & 0  \\
        date picker    & 0  \\
        emoji    & 0  \\
        mmobile participants millenial behavior    & 0  \\
        grid versus item by item    & 0  \\
        Icon versus text navigaiont & 0 \\
        ios picker vs radio button    & 0  \\
        item by item higher number of columns highe    & 0  \\
        item by item higher number of rows higher    & 0  \\
        layout vertical lower item nonresponse than horizontal    & 0  \\
        less item scrolling   & 0  \\
        decremental/incremental scale    & 0  \\
        Modularization    & 0  \\
        radio button alternatives (emoji)    & 0  \\
        one versus two columns    & 0  \\
        auto forwarding    & 0  \\
        Paging versus scrolling    & 0  \\
        Question all at aonce vs section vs one at a time & 0 \\
        Slider & 0 \\
        Slider versus Buttons  & 0 \\
        vertical vs horizotnal orientation &0 \\
        Videp & 0  \\
        Visual analogue scale vs buttons & 0 \\
        Images & 0 \\
		\bottomrule 
	\end{tabular}
	\caption{Overview of the Professional Survey Operators used for more than one survey in the review}
	\label{tab: design}
\end{table}

What are missing categories?
Based on this we identify missing dimensions that could be analyzed in furhter research projects. We cannot rule out that there is already research on this, that was not (yet) published in a scientific fomrat. Especially the presentation at the AAPOR conferece always has interesting research results. 

This is an overview of the design elemnts observed as there is not many overlapping research, so that a meta analysis does not make much sense. There is already some research that aggregate the evicdence and give some deisng recommednation [citation]. These will need updates and can be complemented with the identified missing cateogries.

In this section we analyze the usage of alternative input formats, that are specifically interesting for mobile. These technologies can also be used on pc but are more familiar to the use on mobile devices. Therefore their deployment in mobile surveys is from higher importance. This section is included, as these topics can change the way survey are done from the classical pc format that we are now used to an is wele researched. It will be interesting if it possible to receive additional values from this. We have image taking and voice input as well as paradata that can be sensor data. These give new possibilties in the analysis of the participant and their enivorment. We have possible alternative input dimension voice, images, sensor data and messenger mode (which is not striclty alternativ input mode but should mimic another input mode) . 

These dimension are a new research field and complemented by passive data colleciton, that is not addressed in this review an interesting extension of web surveys which are specially feasible on smartphone devices. They can supplment current reserach [citation] and help to imrpvoe research data collection. We will shortly dicussed the current finding around these fields and give future research recommendations. 

We now give a short overview of the first finding of these dimensions.

There are a lot of further intereseting research like the invitation mode that could be researched in other revioews. 
(Interessant sind andere modes wie Einladung durch app oder push benachcrichtung in pwa)


missing is gamification

How should quesitons be formulated to be good on mobile