In the current buzz on the collection of big data, the importance of classical survey data seems to be overlooked by some, even though the relevance for research and businesses is still high \cite{couper_is_2013, miller_is_2017}. Research based on surveys has to deal with declining response rates \cite{miller_is_2017} and the increasing prevalence of non-probability sampling \cite{couper_is_2013}. Classic approaches for probability sampling like random digit dialling are not working anymore, as the trend goes the use of mobile phones and the transfer of the technique to mobile phones did not show promising results \cite{bucher_exploring_2021}. One exciting approach to counter these trends is to use the increased tendency of smartphone usage \cite{weigold_computerized_2021} by recruiting and operating surveys with these devices. This systematic literature review gives an overview of the research area and identifies trends and research gaps. 

Online surveys are increasingly crucial for survey research and are the dominating mode in most countries \cite{fielding_sage_2017}. They promise cheaper, more flexible and easier to distribute surveys that can implement a higher facet of question types compared to other modes \cite{gummer_does_2019}. More and more online surveys are accessed via mobile devices \cite{de_bruijne_mobile_2014} inducing the need for research on the effect these devices have on online surveys. As the role of tablets is secondary compared to smartphones and their allocation into a group with smartphones is contested \cite{couper_why_2017}, this work will focus only on smartphones. 

There are already existing literature reviews that summarise different dimensions of smartphone survey participants that we will discuss shortly. Mixed modes surveys were already reviewed in \cite{callegaro_mixed-mode_2013} and \cite{deleeuw_mixed-mode_2018}. They give a rather holistic overview without deeper details about the different aspects of smartphone usage in web surveys. \cite{wells_what_2015} gives an unsystematic and cut down an overview of the existing knowledge on mobile participants and data quality. Other works like the article of \cite{couper_why_2017} focus only on one specific aspect as survey duration in this case. A first systematic review was done in \cite{antoun_design_2018}, by summarising research input for mobile survey design. This work offers interesting insights and import design guidelines but is mostly based on conference presentations, white papers and other not peer-reviewed materials. 

This work differs from the previous works in several aspects. Firstly we focus on the research field in general and try to give an exhaustive overview of the literature covering the effect of smartphone participation in web surveys in contrast to the thematic restrictions of existing literature. To ensure we include only high qualitative scientific work, we restrict our work to articles published in peer-reviewed journals or conference proceedings. This aspect and the systematic approach of our work are major distinctive factors ensuring higher objectivity compared to the current work.  

Our contributions are a general up to date overview of the research field, the identification of current research gaps by critically analysing existing research, and the enablement of sophisticated meta-analyses of research topics with the articles identified and the information extracted accessible in \cite{langenbahn_smartphone_2021}. To overview the research field, we defined the following research questions covered in this article.

\begin{itemize}
   \item \textbf{Research Question 1}: Which characteristics of the use of smartphones in web surveys are researched on, and what are research gaps in the field? 
   \item \textbf{Research Question 2}: Which dimension of survey quality of smartphone participation in web surveys are investigated, and which factors lack qualitative research?
   \item \textbf{Research Question 3}: Which elements of online survey design are investigated for adjustments to the use of smartphones, and which aspects are missing in current research?
\end{itemize}

We structure our work into four sections. In the next section, we will present the systematic review methodology and critically discuss potential bias in work. The third section showcases the main results of this article and provides deeper insights into the research field. In the last section, we summarise our findings and give an outlook for future research endeavours. 




