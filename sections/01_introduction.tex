While we live in a world of big data the meaning of survey data did not diminish and is import as ever. There however problems of diminisihing repsonse rates and the dififculty of getting participant. Random digit dialing is in the past based on the mobile reovlution. The question is how to adess. Since mobile is the easiest acess point for many group, it will be important to translate the tool in the mobile age. This slr will analyze the research area on current trends, patterns and missing research, that will be needed. It should serve researcher that want to get an overview of the research field and practioner, that want to get aquianted with the basic needed to understand mobile surveys. 
Based on increasing mobile usage of the internet, this is and will be the most important device for assessing information on web. This means also online surveys a trending and increasingly important format has to adapt to this. What are the specialities

Paragraph about the Importance of research on mobile devices

Paragraph about general introduction of the survey research area

Paragraph about the current research development

Paragraph why we need the systematic literature review

This review will give an overview of the current research field development, critically discuss the current research field, evaluate some dimeniosn in a qualitative meta analyses and give input on future reserach projects.

Paragraph about how we do this systematic literature review

Paragraph about the research questions


Which elements of online survey design are investigated 

\begin{itemize}
   \item \textbf{Research Question 1}: Which characteristics of the use of smartphones in web surveys are researched on and what are research gaps in the field? 
   \item \textbf{Research Question 2}: Which dimension of survey quality of smartphone participation in web surveys are investigated, and which factors lack qualitative research?
   \item \textbf{Research Question 3}: Which elements of online survey design are investigated for adjustments to the use of smartphones, and which aspects are missing in current research?
\end{itemize}

Paragraph about the structure of this work


Besides the use of Big Data [citation], non probability sampling [citaiotn], there is a lot of research going on on smartphones in web surveys.\cite{couper_is_2013}

This section will give an short overview of the research area and where we will position our work. To do reserach with surveys is an very old research area [citaiton], that is an highly used method in sociual science and other research fields. It has already see some changes [citaiton] and is continously develeoping. Newer developments has seen teh importance of survey data reducing [citaiton] while there other arguing that the importqance of surveys are stil high [citaiton]. The acutal development in survey research  is the strong reliance on online surveys as they are cheaper, faster and reach more people [citation]. This trends is continuing and will be most likely persist in the next years [citaiton]. A comparable newer development is the use of mobile devices like smartphones and tablet to access these online surveys [citation]. As these device difer significantly they have to be studied seperately from pc online surveys and their effect on online surveys should be investigted [citation].
Based on increasing mobile usage of the internet, this is and will be the most important device for assessing information on web. This means also online surveys a trending and increasingly important format has to adapt to this. What are the specialities

Tablets and smartphones continue to gain popularity, with a forecasted average shipment of 1.27 million (Tankovska, 2020) and 1.57 billion (O’Dea, 2020) worldwide, respectively, in 2022 \cite{weigold_computerized_2021}

Compared to face-to-face interviews, interviewing respondents online has an appealdue to lower costs and other beneficial side effects (e.g., reducing interview effects, more flexibility,and implementation of visual cues and media applications) \cite{gummer_does_2019}

Given the small number of tablet users and the evidence reviewedabove that tablets behave very similarly to PCs in terms of performance on web surveys, we combinethem with the PC group. \cite{couper_why_2017}

The results show that the spontaneous, unintended mobile attempts to complete online question-naires have significantly increased between 2012 and 2013. In the LISS panel, the share of mobileresponse increased from 3\% in March 2012 to 11\% in September 2013. In the CentERpanel, thespontaneous mobile response increased from 3\%in February 2012 to 16\%in October 2013. \cite{de_bruijne_mobile_2014}


Web participation with smartphone is a subcategory of online surveys in general. There is already a lot fo research on online surveys. While there is already a lot of resarch on online surveys [citation]. shortly summarizing the main findings. There exist already reviews for online surveys for responser rates [citation], gamificaiton, adaption to elderly [citaiton] and further. It could be possible that they will need an update for the newer setup we have, but there seems to be already a lot of work done. While there is not an overview of the literature concerning mobile devices, that this work will try to fill.

THere were already a few review of parts of smartphone participation, that we want to shortly discuss here, to show how our work will be seperated from them. The newer field is to understand mobile web survey particiapnts and thei rbevhaior. There are already some approaches trying to summarizing the state of mobile survey research. Here is what they all missing. 

We have different types of review, there review focussing on data qualilty, focussing on survey design and then review that try to summarize the general develoipment of surveys. Our foucs is on all three, as we will sumamrize the general development of smartphone survey and then focuss on survey design and data quality. 

In his review Couper \cite{couper_is_2013} is reviewing all trends that are tehcnolgocial caused and influencing survey methodology. He mentions mobile surveys as on of three key trends and does give an outlook on the future of survey. This work is rather subjective in the nature and does not aggregate literature in a analysis way. Similair \cite{callegaro_mixed-mode_2013} does only give an short overview of the field and mobile app use and their potential impact on the field.
The work of \cite{toepoel_online_2015} gives only an overview of possivle research in the issue of the journal.Another overview of the researhc field \cite{link_mobile_2014}. 

\cite{deleeuw_mixed-mode_2018} gives an updated view on the mixed mode setting.

\cite{wells_what_2015} gives and overview of resarch knoweldge on Mobile Participants and data quality without an systematic approach to this knoweldge and without criticising. 

There also exist work focussing on one speciifc asect as for example the lenght of web surveye \cite{couper_why_2017}

There is also work on survey design and event a systematic review of design elements \cite{antoun_design_2018}, even though they include all kind of literatuer also only presentation that could reduce the quality of their results. Even though tjis and and their design recommendation are interesitng.


Our work seperates it in that it uses an systematic approach to identifiy relevant literature to reduce the bias in source seleciton. This can serve as a basis for sophisticated meta analysis on various topic where the exist mixed results. Another part is that we only use high quality scientifc sources that were peer reviewed and thus adhree to a specific scineitifc standard. This is not done by most of the exisitng review. Additionally we update the results of the exisitng review to the year 2021, an important thing as mobile surveys are quickly developing. 




