The trend of mobile web surveys partly obsoletes current knowledge about the online survey, and a vast amount of research tries to fill this gap. This systematic literature review gave an overview of the different research fields and pointed out research gaps in smartphone web survey participation. We briefly introduced the topic before we described and discussed the systematic literature review process. Our main results were presented in the third section, where we showcased current research areas and identified several research gaps.  

Our review comprised 86 peer-reviewed articles from the research field of online surveys taken on a smartphone from 2009 to 2021. We identified publication trends upward until 2019 and deep-dived into the two interdependent areas, survey quality and survey design. In the tables of the third section, one can find current research areas and the matching literature. An essential research gap we identified is the missing coverage of the population of different world regions as most of the research is based on data from western Europe and North America. Additionally, we saw that it would be essential to update the results to keep up with the ever-changing technological landscape and adoption behaviour. Other research gaps we found were missing insights on incentivisation of mobile users, gamification in mobile survey design, the usage of mobile technologies like push notifications and the adoption of question formulation guidelines for mobile device surveys. 

This work has several weaknesses that need to be minded when interpreting and using the results. The systematic review methodology is limited by a missing second researcher with sophisticated domain knowledge. The scope of this post-graduate coursework is a limited specified scope, which does not allow for more profound analyses of the different aspects of smartphone survey participation. 

Besides the identified research gaps, other possible research projects result from this work. Another researcher can use this work as a baseline for a more professional review and quality control for literature searches. Another straightforward research project could be to utilise the unused extracted data in \cite{langenbahn_smartphone_2021} for a qualitative and quantitative meta-analysis of the different research aspects to distil well-founded insights for practitioners.

This review made it apparent that smartphone participation in online surveys is a vivid and dynamically developing research area, which is increasingly vital for data collection. Therefore, further work in the research field is highly encouraged.



