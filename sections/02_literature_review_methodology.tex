Before analysing the results of the systematic literature review, we present the review protocol and discuss the strengths and weaknesses of this work. The methodology of the systematic review is largely based on \cite{silva_systematic_2016} and complemented by the work of \cite{xiao_guidance_2017},\cite{petticrew_systematic_2008},\cite{snyder_literature_2019} and \cite{denyer_producing_2009}. We used the eleven steps presented in \cite{silva_systematic_2016} to structure our systematic review.

\subsection{Review Protocol}

The review protocol is structured according to \cite{silva_systematic_2016} and should allow the reader to evaluate the objectivity and quality of the final results (\cite{page_prisma_2021}). The review protocol was iteratively developed, and only the final result is presented here.


\subsubsection{Defining the main research questions of the literature review}

Based on the need for an updated review of the research field concerning web survey participation with smartphones, we developed three research questions that we presented in section \ref{01_introduction}.

\subsubsection{Defining keywords}
\label{subsubsec: Defining search string}

After a first literature search and multiple iterations, we identified the following search string for our literature search: \\

\begin{quote}
(“mobile” OR “phone” OR “tablet” OR “device”) \\

AND ("web" OR "online" OR "digital" OR "browser ")  \\
    
AND ("survey" OR "question*" OR "panel" OR "poll")  \\
    
NOT ("bank*" OR "payment" OR "e-commerce" OR "health" OR "shop*" OR "library" OR "education" OR "sex*" OR "dating" OR "tourism" OR "travel" OR "social network" OR "social media" OR "internet of things" OR "iot" OR "advertise*")\\
\end{quote}

We identified the final combination of terms based on the precision and recall for retrieving relevant entries for the combination of the search terms and search settings. The first three lines should ensure that we find all relevant articles (increase recall), while the last line should ensure that we do not find too many non-relevant records (increase precision).


\subsubsection{Defining search engines}
\label{subsubsec: Defining search engines}

We selected the searched databases based on recommendations of the library of the University of Mannheim and the evaluations in the articles of \cite{pascoe_systematic_2021}, \cite{papaioannou_literature_2010} and \cite{gusenbauer_which_2020}. We complemented the list with the database of the publisher of influential journals in the research area. We excluded databases and articles where we had no free access as a student of the University of Mannheim. If a search engine had multiple databases included, we state the databases selected. The resulting final list comprised the following six databases:

\begin{itemize}
    \item  Web of Science
    \item ProQuest
    \begin{itemize}
        \item Applied Social Sciences Index \& Abstracts (ASSIA)
        \item Business Market Research Collection
        \item Sociological Abstracts
        \item Worldwide Political Science Abstracts
    \end{itemize}
    \item EBSCO
    \begin{itemize}
        \item APA PsycArticles
        \item APA PsycInfo
        \item PSYNDEX Literature with PSYNDEX Test
        \item Communication \& Mass Media Complete
    \end{itemize}
    \item Sage Journal
    \item Oxford Publisher
    \item Tanford Publishing
\end{itemize}

\subsubsection{Search string execution}

Based on the search string in \ref{subsubsec: Defining search string} we conducted the search in the databases mentioned in \ref{subsubsec: Defining search engines} on the 20.12.2021. Additional settings in the search engine that were used when applicable are presented in table \ref{tab: search settings}. The search was conducted by searching in the title, abstract, and keywords. If a combined search of these fields was impossible, we searched only in the abstract.

\begin{table}
	\centering
	\begin{tabular}{ll}
		\toprule
		Settings & Values \\
		\midrule
		Time Horizon & Publications from year 2008 to 2021 \\ 
		Literature Type & Research articles published in peer-reviewed \\
		& proceedings and journals \\
		Research Field & Social Science, Political Science, Business, Economics, \\
		& Computer Science, Psychology, Communication \\
		Language & English \\
		\bottomrule 
	\end{tabular}
	\label{tab: search settings}
	\caption{Additional settings for the search engines of the bibliographic databases}
\end{table} 

\subsubsection{Download and store search results}

The bibliographic details of the search results were downloaded as .ris files by offered download functions or retrieved directly from the site with the bibliographic program Mendeley. The resulting files were saved and managed in Mendeley and Zotero's bibliographic programs. We found 2787 records in the search, which resulted in 1580 non duplicate entries.

\subsubsection{Define inclusion and exclusion criteria}
\label{subsec: Inclusion and Exclusion Criteria}

For the final selection of relevant articles for the systematic literature review, we defined a set of inclusion and exclusion criteria. These criteria are the foundation to determine relevant research articles in the selection stage.

\paragraph{Inclusion Criteria}
\begin{itemize}
    \item Research articles that focus on mobile web surveys
    \item Research articles that explicitly state the role of smartphones in web surveys
    \item Research articles that are in English
    \item Research articles that were published between 2008 and 2021
    \item Research articles that are published in a peer-reviewed journal
    \item Research articles that are published in a peer-reviewed conference proceeding
\end{itemize}

\paragraph{Exclusion Criteria}
\begin{itemize}
    \item Research articles that do not differentiate the effects of different mobile devices like tablets and smartphones
    \item Research articles that use mobile devices only under the supervision of an interviewer
    \item Research articles that include mobile devices in web surveys without specifically analysing their impact
    \item Research articles on mobile forms in general
    \item Research articles focusing on mobile web tests or assessments
    \item Research articles about passive data collection with the smartphone
    \item Research articles that only focus on a small, unrepresentative group in medical trials
    \item Research articles that present specifically developed software
\end{itemize}

\subsubsection{Selection of papers - First and Second stage}

We screened the results of the literature search by title, abstract, and full text based on the in section \ref{subsec: Inclusion and Exclusion Criteria} defined inclusion and exclusion criteria. The title screening resulted in 147 selected records, which were reduced to 104 articles in the abstract screening. We found 12 additional articles in a back and forward citation search, resulting in 116 articles for the full-text screening. After the last screening stage, we identified the final sample of 86 articles. The entire process of the selection of the records can be retraced in figure \ref{fig: Prisma Statement}.

\begin{figure}
    \centering
    \includegraphics[width=\textwidth]{reports/figures/prisma_statement.png}
    \caption{Prisma statment for the systematic literature review in accordance with \cite{page_prisma_2021}}
    \label{fig: Prisma Statement}
\end{figure}

\subsubsection{Extraction of answers related to research questions}

After the selection, the information for analysing the research questions was extracted by a single coder. The information was saved in a spreadsheet that was used for further analysis. Besides the research dimensions, we also coded statistical significance tests' results for uttered hypotheses. The interested reader can refer to all results of the review in the accompanying GitHub Repository (\cite{langenbahn_smartphone_2021}).  


\subsection{Discussion of procedure}

We critically discuss this systematic literature review method to enable the reader to evaluate the objectivity and validity of the presented results. The article was written in postgraduate coursework by a single student. For this, we choose the guideline from \cite{silva_systematic_2016} based on their feasibility for postgraduate students and the possibility to use the guidelines as a single reviewer. Other approaches and procedures as presented in \cite{xiao_guidance_2017}, \cite{petticrew_systematic_2008}, \cite{snyder_literature_2019}, and \cite{denyer_producing_2009} may be more appropriate for a professional setup but were out of the scope for this coursework. 


\subsubsection{Analysis of bias in this work}

We will base our analysis on the bias classes identified in \cite{durach_new_2017}: sample bias, selection bias, within-study bias and expectancy bias.

The class of sample bias divides into retrieval and publication bias, which are both to a certain degree present in this work. Retrieval bias may occur in selecting databases, creating the search strings, and the search settings used. Our choice of databases was based on best practices promoted by the University of Mannheim library, and the results of \cite{pascoe_systematic_2021}, \cite{papaioannou_literature_2010}, and \cite{gusenbauer_which_2020}. However, this selection was limited to the resources available to University of Mannheim students, which excluded access to SCOPUS and some articles hidden behind a paywall. This restriction will have led to a retrieval bias. The search string and settings showed a high recall with only 12 added articles from forward and backward citation search. As we included all important journals and a wide range of databases, it is not probable that we have missed whole clusters of research articles that we could not identify in the citation search. However, the search string has a very low precision, which has led to much ineffectual work, potentially inducing mistakes in the selection. Another possible retrieval and publication bias is that this review only used articles published in a peer-reviewed scientific medium. Our citation search has shown that many impactful scientific works are not published in a peer-reviewed journal or proceeding. These works are most of the time presented at the AAPOR Annual Conference, at the CASRO Online Research Conference or in the editor reviewed journal Survey Practice. An enhancement of this work could include these publications and grey literature in the review to get a more comprehensive overview of the research field. However, this will induce much work in revising the quality of the publication, which was not feasible in this work.

Selection biases partition into inclusion and selector biases, where an inclusion bias is highly likely in this work and the selector biases relatively insignificant. An inclusion bias implies not adequately defined inclusion and exclusion criteria. This bias will most likely be present in this work as no research experts for mobile web surveys designed the criteria. The selector bias is not significant as the setup with only one person creating inclusion and exclusion rules and selecting the articles does not offer much potential for inconsistencies.

The within-study bias describes errors in extracting information from the selected articles. Only one person extracted information from the data, so we cannot guarantee the extraction quality by measurement like the inter-annotator agreement. This setup leads to an underestimated within-study bias for this work.

The expectancy bias is negligible in this work. The author of this work does not have any prior professional knowledge about mobile web surveys or personal interests that could lead to any preconceptions.

\subsubsection{Future research}
Finally, we want to discuss how another researcher could advance this review on the level of a scientific publication. We have already identified methodological weaknesses in the last section that need to be fixed. Additionally, we will shortly discuss the range of topics in this work. To avoid technical mistakes, the replication of this work would need at least one experienced researcher with expertise in mobile web surveys to use their domain knowledge to avoid the identified technical biases. Additionally, another researcher is required to increase the robustness of the selection and information extraction. With more resources, it would be possible to reduce the retrieval bias by including the database SCOPUS and gaining access to paid articles. Further improvements could be the fine-tuning of the search string and the inclusion of the characteristics of different search engines. To further avoid publication bias, one could include qualified scientific literature not published in peer-reviewed journals or proceedings. As this article is limited in its scope, we could not perform sophisticated meta-analyses and quantitative summaries but had to constrain ourselves to a general overview of the research field. In the following research project, one should concentrate on one aspect of smartphone web surveys and present the results of quantitative meta-analyses. To summarise, we can state that the present work does not wholly fulfil all scientific standards but functions as a basis for further research work by reducing the workload and serving as a quality control medium. 