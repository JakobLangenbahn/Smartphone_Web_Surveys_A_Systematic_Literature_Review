This section will give an short overview of the research area and where we will position our work. To do reserach with surveys is an very old research area [citaiton], that is an highly used method in sociual science and other research fields. It has already see some changes [citaiton] and is continously develeoping. Newer developments has seen teh importance of survey data reducing [citaiton] while there other arguing that the importqance of surveys are stil high [citaiton]. The acutal development in survey research  is the strong reliance on online surveys as they are cheaper, faster and reach more people [citation]. This trends is continuing and will be most likely persist in the next years [citaiton]. A comparable newer development is the use of mobile devices like smartphones and tablet to access these online surveys [citation]. As these device difer significantly they have to be studied seperately from pc online surveys and their effect on online surveys should be investigted [citation].
Based on increasing mobile usage of the internet, this is and will be the most important device for assessing information on web. This means also online surveys a trending and increasingly important format has to adapt to this. What are the specialities

Tablets and smartphones continue to gain popularity, with a forecasted average shipment of 1.27 million (Tankovska, 2020) and 1.57 billion (O’Dea, 2020) worldwide, respectively, in 2022 \cite{weigold_computerized_2021}

Compared to face-to-face interviews, interviewing respondents online has an appealdue to lower costs and other beneficial side effects (e.g., reducing interview effects, more flexibility,and implementation of visual cues and media applications) \cite{gummer_does_2019}

Given the small number of tablet users and the evidence reviewedabove that tablets behave very similarly to PCs in terms of performance on web surveys, we combinethem with the PC group. \cite{couper_why_2017}

The results show that the spontaneous, unintended mobile attempts to complete online question-naires have significantly increased between 2012 and 2013. In the LISS panel, the share of mobileresponse increased from 3\% in March 2012 to 11\% in September 2013. In the CentERpanel, thespontaneous mobile response increased from 3\%in February 2012 to 16\%in October 2013. \cite{de_bruijne_mobile_2014}


Current topics survey research is focussing on [citation] and so on.


While there is already a lot of resarch on online surveys [citation]. shortly summarizing the main findings.



The newer field is to understand mobile web survey particiapnts and thei rbevhaior. There are already some approaches trying to summarizing the state of mobile survey research. Here is what they all missing. 