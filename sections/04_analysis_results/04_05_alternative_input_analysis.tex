In this section we analyze the usage of alternative input formats, that are specifically interesting for mobile. These technologies can also be used on pc but are more familiar to the use on mobile devices. Therefore their deployment in mobile surveys is from higher importance. This section is included, as these topics can change the way survey are done from the classical pc format that we are now used to an is wele researched. It will be interesting if it possible to receive additional values from this. We have image taking and voice input as well as paradata that can be sensor data. These give new possibilties in the analysis of the participant and their enivorment. We have possible alternative input dimension voice, images, sensor data and messenger mode (which is not striclty alternativ input mode but should mimic another input mode) . 

These dimension are a new research field and complemented by passive data colleciton, that is not addressed in this review an interesting extension of web surveys which are specially feasible on smartphone devices. They can supplment current reserach [citation] and help to imrpvoe research data collection. We will shortly dicussed the current finding around these fields and give future research recommendations. 

We now give a short overview of the first finding of these dimensions.

There are a lot of further intereseting research like the invitation mode that could be researched in other revioews. 
(Interessant sind andere modes wie Einladung durch app oder push benachcrichtung in pwa)