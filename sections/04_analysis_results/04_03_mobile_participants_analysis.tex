In this section we analyze the participants and possible participants of mobile surveys. There are two important thinks here to consider. One ist see who is taking surveys on their mobile phone in the moemnt and if there is a ubstantial bias induced by this. Second we check who is willing to take surveys on a smartphone for future endevaours. Both is helpful to understand problems like coverage bias and simialr, for conducting mobile surveys.

We will start with a small overview of the research areas that are present in this work about the participants. After this we will nanalyze specific point and aggregate the results in a qualitative way.



\begin{table}
	\centering
	\begin{tabular}{ll}
		\toprule
		Data Quality Topic  &  Research Articles \\
		\midrule
        mobile participants coverage bias \& mobile participants representative not    & 0  \\
        mobile participants education    & 0  \\
        mobile participants income higher   & 0  \\
        mobile participants less alone    & 0  \\
        mobile participants less trust in confidentiality    & 0  \\
        mmobile participants millenial behavior    & 0  \\
        mobile participants more distracted    & 0  \\
        mobile participants more female
        mobile participants more mobile    & 0  \\
        mobile participants more moving    & 0  \\
        mobile participants more multitasking    & 0  \\
        mobile participants more progressive    & 0  \\
        mobile participants more urban    & 0  \\
        mobile participants more willing passive data collection    & 0  \\
        mobile participants more stranger present    & 0  \\
        mobile participants multiple browser sessions more    & 0  \\
        mobile participants sensitivity perceived higher    & 0  \\
        mobile participants social class higher    & 0  \\
        mobile participants younger    & 0  \\
		\bottomrule 
	\end{tabular}
	\caption{Overview of the Professional Survey Operators used for more than one survey in the review}
	\label{tab: author}
\end{table}


We have a few topics with multiple articles focussing on it, where we will do a qualitative meta analyiss. The first one is the age of mobile participants. There is strong evidence that people that decide to access online surveys via smartphone are younger than people accessing with a PC. 13 Survey could support this hypothesis and only one could not support this. This seems to be fitting to the general trend of higher smartphone usage by younger adults [citation.]
It was also found in the survey on willgness that [citation]

When analysing the gender of the smartphone participants in web surveys we have mixed results 8 surveys could support the hypothesis that female use smartphones more often to access online, 5 could not support the hypothesis and two surveys supported the hypothesis that more male user used smartphones to access surveys. There seems to be mixed results for this dimensions. While the evidence indicated more female users this cannot be said with statisitcal security. We did not identifiy strong trends regarding countries or target group. As the last survey was operated in 2017 these results would need an update anyways to update to modern standards. General results about smartphone use of male vs female [Citation]
It was also found in the survey on willgness that [citation]


Another dimension we analyze is the education level of people using smartphones to access online survey. We have four surveys not supporting the hypothesis and 2 survey not supporting this theory. We do not have enough data to find patterns in the operation terms of these surveys. It is not clear if this effect is still relevant, as the last survey was operated in 2015.
It was also found in the survey on willgness that [citation]


There are a lot more dimensions that we could possible analysises given more time and more resources. An systematic review only on participants would be an interesting research project. Especially the changes in this will be interesting depending on future channgees in the use of smartphones espeically for elderly [citation.]

