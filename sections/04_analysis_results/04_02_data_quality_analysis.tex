Simple meta-analysis of survey data comparing different years is not possible based on a tremendous shift in the smartphone and mobile phone enviroemnt, this lead to the effect that the population changed and the gegenebenheiten

Define all categories that can be compared

There is a lot of research going on data quality as we could see in the previous figure. There a a lot of data quality issues controlled as we can in figure x. If we compare this with the quality indicators in (Citiation of an overview of data quality indicators) we see that currently research on is misisng. as this is more complciuated more sophistiacted tehcnolgies would need to be used. 



We will focus our analyis on the most often used measures and compare their results. For this you can refere to the tables, where you see the results and can see what happend.


- Breakoff/Completion rate

- Completion time 

- Missing Items/Non response

- Straightlining
- Primacy effect

- Non Substantial Answer
- Text Anwser

Discussion of results based on the quality indicators and 

We have to view this results as temporary snapshot of the behavior, as our behaviour around mobile device is changing based on a rapidly developing development of the technology with ever new helps, large screen size and more learning effects for various groups (children, teenager, Adults, older population. As well as new design insights, that help to reduce the negative effext and coild also increase positive effects of mobile technologies.

Table for these dimensions

comepltion time

mobile mising item

straightlining

text answer short

text answer avoid

Summarize results of this 

FIGURE OF Data Quality Issues Discussed


Statistical analysis of the results from the various researches


As we can see data quality is not fully impaired but sitll have 

Discuss other topics shortly and maybe check what is missing
Bias
- Response order effects
- Social desirability
- Non substantive responses
- lenght of open. answers
- Breakoff rate
- Item non-differentiation
- Item Nonresponse
- Number of responses in check-all-that-apply question
- Robustness
- Interview duration
- straightlining,
- choice of left-aligned options in horizontal scales
- Completion rate
- Willigness to answer complex and open-ended questions
- Non differentiation
- Primacy effect
- Mean number of answers in check-all-that-apply
- Duration
- participation rates
- abandon rates
- check failure
- nondifferentation
- item missing data
- non substantibe responses
- Propensity score matching
- acquiescence tendency
- reaction time to survey invitation
- survey transmission
- reliability and vaidity of scale responses
- Satisfcing
- Neutral and extreme responding 
- mitvated misreponding
 - Research messenger

Reporting on sensitive items


Privacy
