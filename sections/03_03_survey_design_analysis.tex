Which elements of online survey design are investigated for adjustments to the use of smartphones, and which aspects are missing in current research?

The previously discussed survey quality dimensions not solely depend on the device used to access the survey but also on the survey design and the adjustments made for smartphone usage. We identified the significant impact of device optimisation on most dimensions identified in the analysis of survey quality. The topic of mobile survey design was already covered by \cite{antoun_design_2018} in a systematic review. While this review included many, not peer-reviewed publications, we will focus our overview on only peer-reviewed articles. 

Table \ref{tab: design} showcases the identified design dimensions covered in current research projects and the matching articles of our review. We combined mobile design decisions, alternative input forms and invitation modes into one category. 

\begin{table}
	\centering
	\begin{tabular}{ll}
		\toprule
		Design aspect  &  Research Articles \\
		\midrule
		Mobile optimisation impact & \cite{arn_evaluation_2015, revilla_open_2016}\\
		Modularisation & \cite{antoun_design_2018, callegaro_mixed-mode_2013}\\
    	& \cite{toepoel_modularization_2018, bansal_shorter_2017}\\
    	& \cite{mason_effect_2019}\\
    	Paging & \cite{de_bruijne_improving_2014, mavletova_mobile_2014}\\
        Vertical versus horizontal orientation & \cite{de_bruijne_improving_2014}\\
        Grid and matrix questions & \cite{revilla_experiment_2017, revilla_comparing_2018}\\
    	& \cite{mavletova_grid_2018, revilla_testing_2018}\\
    	& \cite{tourangeau_web_2017}\\
        Item by item questions & \cite{revilla_experiment_2017, revilla_comparing_2018}\\
    	& \cite{mavletova_grid_2018, revilla_testing_2018}\\
    	& \cite{tourangeau_web_2017, grady_what_2019}\\
        Slider & \cite{antoun_effects_2017, maineri_slider_2021}\\
    	& \cite{toepoel_sliders_2018}\\
        Radio buttons & \cite{olmsted-hawala_optimal_2018, cernat_radio_2019}\\
        Visual analogue scale & \cite{toepoel_sliders_2018}\\
        Navigation with & \cite{revilla_comparing_2018, olmsted-hawala_optimal_2018}\\
        text, buttons or icons & \cite{lugtig_recruiting_2019, revilla_testing_2018}\\
        Image and Video display & \cite{mendelson_displaying_2017, trubner_effects_2020}\\
    	Voice input & \cite{revilla_improving_2021, revilla_testing_2020}\\
    	Image input &\cite{bosch_answering_2019}\\
    	Messenger input & \cite{toepoel_probing_2021}\\
    	Sensor data input & \cite{hohne_motion_2020, hohne_surveymotion_2019}\\
    	Emoji  use & \cite{bacon_how_2017, bosch_using_2021}\\
    	SMS invitation & \cite{de_bruijne_comparing_2013, mavletova_data_2013}\\
    	& \cite{bucher_exploring_2021, de_bruijne_improving_2014}\\
    	& \cite{schlosser_mobile_2018, mavletova_mobile_2014}\\
    	Mail invitation & \cite{lugtig_recruiting_2019}\\
    	Other & \cite{antoun_effects_2017, wang_experimentation_2017}\\
    	& \cite{krebs_exploring_2021, mavletova_grouping_2016}\\
    	& \cite{maineri_slider_2021, revilla_testing_2018}\\
		\bottomrule 
	\end{tabular}
	\caption{Overview survey design dimensions covered in current research.}
	\label{tab: design}
\end{table}

Based on the analysis of the design dimensions, we identified three vital research gaps that need the research community's attention. The first one treats the aspect of gamification in mobile web surveys to increase survey participation and data quality. \cite{keusch_review_2015} already reviewed this topic for online surveys in general. The collected insights need to be updated and adapted to mobile devices and their characteristics. Another feature of mobile devices not yet covered in research is push notifications, which could be used as an alternative invitation mode. Their success in other areas was already proven \cite{stroud_effects_2019, pham_effects_2016} and should be tested in the context of mobile web surveys. Another essential aspect of survey design is the formulation of survey questions that can heavily influence the participant \cite{tourangeau_psychology_2000, groves_survey_2009}. Since text representation and reading behaviour differ between personal computers and smartphones \cite{liu_reading_2016}, it could be necessary to adapt best practices for question formulations for the mobile user. Research on these design dimensions will help further to increase the quality of smartphone online survey responses.  