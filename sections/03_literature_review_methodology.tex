Paragraph about the methodology applied to the review and which review protocl we are following

\subsection{Review Protocol}

Explanation about the protocol

\subsubsection{Defining the main research questions of the literature review}



\begin{itemize}
    \item RQ1: 
    \item RQ2: 
    \item RQ3: 
\end{itemize}

\subsubsection{Defining keywords}
\label{subsubsec: Defining keywords}



\begin{table}
	\caption{List of considered keywords}
	\centering
	\begin{tabular}{ll}
		\toprule
		Category & Keywords \\
		\midrule
		Data-Driven & "Machine Learning", "Data-Driven", "Big Data", \\
		& "Demand Forecast" \\
		Inventory Management & "Inventory Management", "Production Quantity" \\
		& "Order Quantity", "Newsvendor",  \\
		\bottomrule 
	\end{tabular}
	\label{tab: keywords}
\end{table}

\subsubsection{Defining search string}
\label{subsubsec: Defining search string}



\noindent ("Machine Learning" OR "Data-riven") AND ("Inventory Management" OR "Newsvendor")


\subsubsection{Defining search engines}
\label{subsubsec: Defining search engines}

Which databases did we use why?

\begin{itemize}
    \item Web of Science
\end{itemize}


\subsubsection{Search string execution}

When did we execute the search

\begin{table}
	\caption{Additional settings for the search engines of the bibliographic databases}
	\centering
	\begin{tabular}{ll}
		\toprule
		Settings     & Values \\
		\midrule
		Time Horizon & Publications from year 2008 to 2021 \\ 
		Literature Type & Research Articles published in Proceedings, Journals, Conferences \\
		Research Field & Business, Political Science\\
		\bottomrule 
	\end{tabular}
	\label{tab: search settings}
\end{table} 

\subsubsection{Download and store search results}

How did we download the results

\subsubsection{Define inclusion and exclusion criteria}

Definition of inclusion and exclusion criteria

\paragraph{Inclusion Criteria}
\begin{itemize}
    \item Research articles that are in English
    \item Research articles that were published between 2008 and 2021
    \item Research articles that are published in a journal
    \item Research articles that are published in a proceeding/conference
\end{itemize}

\paragraph{Exclusion Criteria}
\begin{itemize}
    \item Research articles about passive data collection
\end{itemize}

\subsubsection{Selection of papers - First and Second stage}

Explanation of selection process with prisma statement


\subsubsection{Extraction of answers related to research questions}

Explanation of the encoding and the encoding process. Explanation for not clear codings. 

\begin{table}
	\caption{Overview of extracted variables from the selected literature}
	\centering
	\begin{tabular}{ll}
		\toprule
		Dimension & Explanation \\
		\midrule
		Category & Category of the topics of the article \\
		\bottomrule 
	\end{tabular}
	\label{tab: review dimensions}
\end{table}


If multiple years, then coded the earliest year of survey taking


If multiple Topics the main topic was chosen



\subsection{Discussion of procedure}

Critical dicussion of strenghts and weaknesses of the work. 

\subsubsection{Analysis of bias in this work}

We will structure this analysis based on the overview in \cite{durach_new_2017}. 

The first class of possible bias are sample biases, which divides into retrieval and publication biases. Bias in our retrieval process may happen in the selection of databases, the creation of the search strings and the search settings used. 

The next category in \cite{durach_new_2017} is selection biases which partition into inclusion and selector biases. An inclusion bias happens if the inclusion and exclusion criteria are not adequately defined. 

The third category is the within-study bias, which describes errors in extracting information from the selected paper. 

The expectancy bias is the last category in \cite{durach_new_2017}, which is negligible in this work because the author did not have any profound prior knowledge about the research field and did therefore not have any preconceptions.

Discussion about the grey literature

\subsubsection{Future Steps}
Possible next steps to validate the results and how to repeat this project.
